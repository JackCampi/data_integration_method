% =======================
%  Two-column paper style
% =======================
\documentclass[11pt,a4paper]{article}

% ---------- Language & encoding ----------
\usepackage[T1]{fontenc}
\usepackage[utf8]{inputenc}
\usepackage[english,spanish]{babel}

% ---------- Geometry ----------
\usepackage[a4paper,margin=2.3cm]{geometry}

% ---------- Fonts ----------
\usepackage{newtxtext,newtxmath} % Times-like academic font
\usepackage{microtype}

% ---------- Math ----------


% ---------- Figures / tables ----------
\usepackage{graphicx}
\usepackage{booktabs}
\usepackage{caption}
\usepackage{subcaption}

% ---------- Links / bibliography ----------
\usepackage[hidelinks]{hyperref}
\usepackage[numbers,sort&compress]{natbib}

% ---------- Section formatting ----------
\usepackage{titlesec}
\titleformat{\section}{\large\bfseries}{\thesection}{0.6em}{}
\titleformat{\subsection}{\normalsize\bfseries}{\thesubsection}{0.6em}{}
\titlespacing*{\section}{0pt}{1.0ex}{0.8ex}
\titlespacing*{\subsection}{0pt}{0.8ex}{0.6ex}

% ---------- Header ----------
\usepackage{fancyhdr}
\pagestyle{fancy}
\fancyhf{}
\lhead{\small Data integration for kidney transplantation}
\rhead{\small \thepage}
\renewcommand{\headrulewidth}{0.4pt}

% ---------- Abstract box ----------
\usepackage{framed}
\setlength{\FrameSep}{8pt}

% ---------- No indentation ----------
\setlength{\parindent}{0pt}
\setlength{\parskip}{4pt}

% ==================================
%  Title
% ==================================
\title{\Large \textbf{Data Integration Methods and Application to Kidney Transplantation}\\
\vspace{0.3em}\normalsize Final Project -- Mathématiques Appliquées}

\author{
Prénom NOM\thanks{École Centrale de Nantes, MATHAPPLI} \\
\small \texttt{prenom.nom@ec-nantes.fr}
}

\date{\today}

\begin{document}

% ---------- Two-column title block ----------
\twocolumn[
\maketitle
\vspace{-1.2em}

\begin{center}
\small
\textbf{Keywords:} data integration, multi-view learning, transplantation, prediction
\end{center}

\vspace{0.8em}

\begin{framed}
\noindent\textbf{Abstract.}
Dans ce rapport, nous étudions les méthodes d’intégration de données hétérogènes
(cliniques, génétiques, biologiques) dans un cadre d’apprentissage supervisé.
L’objectif est de construire un prédicteur du rejet humoral après transplantation
rénale. Après une synthèse de la littérature, nous appliquons une méthode
d’intégration spécifique et évaluons ses performances sur des données réelles.
\end{framed}

\vspace{1.2em}
]

% ==================================
%  Content
% ==================================
\section{Introduction}
La transplantation rénale est un traitement de référence de l’insuffisance rénale
terminale. Cependant, le rejet humoral demeure une cause majeure d’échec du greffon.
L’intégration de données multi-sources constitue un levier prometteur pour améliorer
les modèles prédictifs \citep{ghojogh2021rkhs}.

\section{Contexte et données}
\subsection{Sources de données}
Description des données cliniques, biologiques et génétiques utilisées, ainsi que
les étapes de pré-traitement.

\subsection{Problème de prédiction}
Définition de la variable cible, horizon temporel, métriques d’évaluation.

\section{Méthodes d’intégration de données}
\subsection{Synthèse bibliographique}
Early integration, late integration, intermediate integration, multi-view learning,
kernel methods, CCA, PLS, etc.

\subsection{Méthode retenue}
Description détaillée de la méthode choisie et justification.

\section{Résultats}
\subsection{Protocole expérimental}
Validation croisée, modèles de référence, réglage des hyperparamètres.

\subsection{Résultats quantitatifs}

\begin{table}[h]
\centering
\caption{Performances des modèles}
\begin{tabular}{lcc}
\toprule
Modèle & AUC & Accuracy \\
\midrule
Clinique seul & 0.71 & 0.68 \\
Intégration multi-vues & 0.79 & 0.74 \\
\bottomrule
\end{tabular}
\end{table}

\section{Discussion}
Analyse critique des résultats, limites de l’étude, interprétabilité.

\section{Conclusion et perspectives}
Résumé des contributions et pistes futures.

% ==================================
%  Bibliography
% ==================================
\bibliographystyle{unsrtnat}
\bibliography{references}

\end{document}
