% =======================
%  Two-column paper style
% =======================
\documentclass[11pt,a4paper]{article}

% ---------- Language & encoding ----------
\usepackage[T1]{fontenc}
\usepackage[utf8]{inputenc}
\usepackage[english]{babel}

% ---------- Geometry ----------
\usepackage[a4paper,margin=1.5cm]{geometry}

% ---------- Fonts ----------
\usepackage{newtxtext,newtxmath} % Times-like academic font
\usepackage{microtype}

% ---------- Math ----------


% ---------- Figures / tables ----------
\usepackage{graphicx}
\usepackage{booktabs}
\usepackage{caption}
\usepackage{subcaption}

% ---------- Links / bibliography ----------
\usepackage[hidelinks]{hyperref}
\usepackage[numbers,sort&compress]{natbib}

% ---------- Section formatting ----------
\usepackage{titlesec}
\titleformat{\section}{\large\bfseries}{\thesection}{0.6em}{}
\titleformat{\subsection}{\normalsize\bfseries}{\thesubsection}{0.6em}{}
\titlespacing*{\section}{0pt}{1.0ex}{0.8ex}
\titlespacing*{\subsection}{0pt}{0.8ex}{0.6ex}

% ---------- Header ----------
\usepackage{fancyhdr}
\pagestyle{fancy}
\fancyhf{}
\lhead{\small Data integration for kidney transplantation}
\rhead{\small \thepage}
\renewcommand{\headrulewidth}{0.4pt}

% ---------- Abstract box ----------
\usepackage{framed}
\setlength{\FrameSep}{8pt}

% ---------- No indentation ----------
\setlength{\parindent}{10pt}
\setlength{\parskip}{4pt}
\setlength{\columnsep}{0.8cm}

% ==================================
%  Title
% ==================================
\title{\Large \textbf{Data Integration Methods and Application to Kidney Transplantation}\\
\vspace{0.3em}\normalsize Final Project -- Mathématiques Appliquées}

\author{
Angel ROMERO, Ana María PINZÓN
\thanks{École Centrale de Nantes, MATHAPPLI} \\
\small {École Centrale de Nantes}
\small {MATHAPPLI}
}

\date{\today}

\begin{document}


% ---------- Two-column title block ----------
\twocolumn[
\maketitle
\vspace{-1.0em}

\noindent\rule{\linewidth}{1.2pt}

\vspace{0.2em}
%\begin{center}
\small
\textbf{Keywords:} data integration, multi-view learning, transplantation, prediction
%\end{center}

%\vspace{0.2em}
% \begin{framed}
% \noindent\textbf{Abstract.}
% Dans ce rapport, nous étudions les méthodes d’intégration de données hétérogènes
% (cliniques, génétiques, biologiques) dans un cadre d’apprentissage supervisé.
% L’objectif est de construire un prédicteur du rejet humoral après transplantation
% rénale. Après une synthèse de la littérature, nous appliquons une méthode
% d’intégration spécifique et évaluons ses performances sur des données réelles.
% \end{framed}

\vspace{1.2em}
]

% ==================================
%  Content
% ==================================

\section{Introduction}


% Pb amount of data
Predictive statistical models have not been used a lot historically in the medical field due to the heterogeneity of the biological data. This means, laboratories obtain different kind of informations from the same individu such as xray images, molecular data, clinical data, etc. This poses a problem for the unification in a single model, since the data structures are very different in nature. To sum there is not great amounts of data available because there is no standarization in many lab procedures, and also because of the cost of collecting data, and even the ethical issues that can arise from collecting data in the medical field. entre otros. This presents a barrier for models which need a lot of observations to be statistically valid in the sense that they assume a huge amount of individuals to make good inferences.

% Pb multi-modal data
In this project we will work with data such as clinical data, demographic data, molecular data, etc. The problem here is how can we relate this variables if they do not live in the same spaces. This a problem of multi-modal data integration.

% Multi-omics data
Latest technologies in biology labs allow for molecular data collection at levels we never had before. Nowadays, we talk about multi-omics data, which observes multiple biological phenotypes by levels of deepness from de the genetics to the phenomics (observable disfunction).

% Possible discussion for improving pred. due to amount of info on an individu
Another difficulty lies in the fact that there are not a lot of individuals in the data base but we do have access to information of different types for the same individual. We would like to see if we can get better predictions correctly relating this variables between them. (Discusion)

% Problem description
For this project we are gonna discuss different ways of integrating multi-modal data (taking into account their nature) in a prediction model and test them to see if we can get relatively good predictions for our specific problem of the rejection humoralafter kidney transplantation.

% Other possible questions
Why do we need data integration tecniques? 
Naive approach :  Just having every variable as a simple covariate? 
Limitations (are they mathematical or simply it is not coherent with the problem?)
If its mathematical, why? (sum of distribution laws, ...)
If its coherence (Different type of data, correlation and causality,...)

% We still need to add a bit of the kidney transplant issue to contextualize the multi-modal problem in the medical field

\section{Data integration methods}

\subsection{First approach (naive?)}
Linear combination of the variables

Normalizing and suming everything

Simply using them all in the same data base?

Proyection in one common space

\subsection{Modern methods and research}
Kernel methods 

Others?

\subsubsection{Kernel methods}

% Kernel methods
A kernel $K$ is a mapping which takes two elements from an arbitrary space and assigns them a real number ($K : \mathcal{X} \times \mathcal{X} \longrightarrow \mathbb{R}$). It can be seen as a "similarity" function. It is symmetric since the resemblance from one element $x_i$ to $x_j$ is exactly the same as the resemblance of $x_j$ to $x_i$. The kernel function can also be seen as the mapping $\phi : \mathcal{X} \longrightarrow \mathcal{F} $ which satisfies 

$$
k(x,x')=<\phi(x),\phi(x')>_\mathcal{F}
$$

The kernel matrix or Gram matrix $ \mathbf{K} \in \mathbb{R}^{n \times n} $ is the symmetric matrix $K_{ij}  = k(x_i,x_j)$ for $i=1,2,...,n$, which must be positive semi-definite \cite{mva_kernels_2022}. 

Kernel methods are algorithms which take the kernel function as an input. The idea behind the kernel methods is to map the data into higher-dimensional spaces where it is nicer to do machine learning. Or, in other words, the points are mapped into a space where they are more easily separable. The interesting thing is that in this new space the points are now represented by their kernel function. 

% Heterogeneous data
This approach represents several advantages for our problem since it doesn't makes any assumtions regarding the type of data, allowing us to map elements from any general space. For example, our points can be real numbers, chains of characters, images, etc. This already starts to look good for our heterogenious type of data. Still, the problem of the integration of the data persists. 

Fortunately, the kernel methods have an approach for this. Multiple Kernel Learning (MKL) were conceived to attack this sort of problems. For this, a kernel matrix is computed for each feature and then they are combined into a final kernel matrix by minimizing an objective function.\cite{Baiao2025}. One advantage for MKL methods is that the final kernel matrix can be used for downstream analysis, supporting the analyses of influencing elements in the prediction \cite{Baiao2025}.

In general, using multiple kernels instead of a single one boosts the prediction capacity of the model. On the other hand, combining kernels in a non-linear way apperntly has better results whereas when combibibg complex Gaussian kernels, linear methods are more reasonable\cite{gonen2011multiple}. 








\section{Discussion}

% How can our model trace influencing elements to give useful retour?


Rigourus enough?

Precission? 

Risk : How much risk would be tolerable for decision making in the 
medical field?

Interpretability? In the medical field, the models have to be precise and have an
accurate argument for the prediction (i.e. which indicator and why is it having
that weight on the decision taken by the model)
Finally, in practice an important question would be, how implied should be
the doctor in the decision? Should it be a model completely autonomous? 





% ======================================================================
% Ideas del gato utiles sobretodo para la segunda parte del proyecto

% \section{Introduction}
% La transplantation rénale est un traitement de référence de l’insuffisance rénale
% terminale. Cependant, le rejet humoral demeure une cause majeure d’échec du greffon.
% L’intégration de données multi-sources constitue un levier prometteur pour améliorer
% les modèles prédictifs \citep{ghojogh2021rkhs}.

% \section{Contexte et données}
% \subsection{Sources de données}
% Description des données cliniques, biologiques et génétiques utilisées, ainsi que
% les étapes de pré-traitement.

% \subsection{Problème de prédiction}
% Définition de la variable cible, horizon temporel, métriques d’évaluation.

% \section{Méthodes d’intégration de données}
% \subsection{Synthèse bibliographique}
% Early integration, late integration, intermediate integration, multi-view learning, kernel methods, CCA, PLS, etc.

% \subsection{Méthode retenue}
% Description détaillée de la méthode choisie et justification.

% \section{Résultats}
% \subsection{Protocole expérimental}
% Validation croisée, modèles de référence, réglage des hyperparamètres.

% \subsection{Résultats quantitatifs}


% \begin{table}[h]
% \centering
% \caption{Performances des modèles}
% \begin{tabular}{lcc}
% \toprule
% Modèle & AUC & Accuracy \\
% \midrule
% Clinique seul & 0.71 & 0.68 \\
% Intégration multi-vues & 0.79 & 0.74 \\
% \bottomrule
% \end{tabular}
% \end{table}

% \section{Discussion}
% Analyse critique des résultats, limites de l’étude, interprétabilité.

% \section{Conclusion et perspectives}
% Résumé des contributions et pistes futures.



% ==================================
%  Bibliography
% ==================================
\bibliographystyle{unsrtnat}
\bibliography{references}

\end{document}
